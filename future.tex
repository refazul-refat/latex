\section{Future Works}

Intel and AMD proposed an extension to the x86 instruction set named AES-NI in 2008 \citep{aesni}. This extension added 7 new instructions to the instruction set with a view to abstracting the encryption process. For example, AESENC instruction would perform one round of an AES encryption flow. A performance analysis showed that AES-NI has an increase in throughput from approximately 28.0 cycles per Byte to 3.5 cycles per Byte \citep{aesni}. Such initiatives are really appreciable, but these sort of instructions are not suitable for RISC processors.\\

Proposing a 4 times slower solution to AES implementation in pursuit of eliminating cache attack threats might not be the best way to come around. There is no point of concerning about size overhead because two 32 bit integers can even be stored in registers these days. The real bottleneck is the inverse mapping function that processes on these integers to get the correct offset to read from a perplexed lookup table. This function is called as many as 16 times during each round. Reduction of even one operation in this function can bring about a visible performance improvement. Creating a new instruction for this function can maximize the improvemnt.\\
