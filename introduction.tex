\section{Introduction}

Out of the five candidate algorithms (MARS \cite{mars1998proposal}, RC6, Rijndael \cite{rijndael1998proposal}, Serpent \cite{serpent1998proposal}, and Twofish \cite{twofish1998proposal}) for the AES, Rijndael was pronounced as a new standard on November 26, 2001 as FIPS PUB 197 \cite{fips}. The security barrier of Rijndael cipher is so strong that a cryptographic break is infeasible with current technology. Even Bruce Schneier, a developer of the competing algorithm Twofish, admired Rijndael cipher in his writing, ``I do not believe that anyone will ever discover an attack that will allow someone to read Rijndael traffic" \cite{admire}.\\

A brute force method would require $2^{128}$ operations for the full recovery of an AES-128 key. However, partial information leaked by side-channels can tear down down this complexity to a very reasonable level. Side-channel attacks do not attack the underlying cipher; they rather attack implementations of the cipher on systems that inadvertently leak data. Ongoing research in the last decade has shown that the information transmitted via side-channels, such as execution time \cite{spadavecchia2006network}, computational faults \cite{boneh}, power consumption \cite{kocher} and electromagnetic emissions \cite{em, em2, scards}, can be detrimental to the security of Rijndael \cite{daemen2002design} and other popular ciphers like RSA \cite{mit}.\\

We will be primarily focusing on the vulnerability of Rijndael AES to side-channel cache-timing attack noticing how simple cache misses can lead to dire consequences. The prevention is not trivial; there exists a trade-off between performance and degree of multiprogramming. We will present a concept that might balance this trade-off by introducing a little memory overhead. The biggest advantage could be that even though the process is arbitrary, it maintains the integrity of the AES.\\
